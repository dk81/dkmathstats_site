\documentclass[]{article}
\usepackage{lmodern}
\usepackage{amssymb,amsmath}
\usepackage{ifxetex,ifluatex}
\usepackage{fixltx2e} % provides \textsubscript
\ifnum 0\ifxetex 1\fi\ifluatex 1\fi=0 % if pdftex
  \usepackage[T1]{fontenc}
  \usepackage[utf8]{inputenc}
\else % if luatex or xelatex
  \ifxetex
    \usepackage{mathspec}
  \else
    \usepackage{fontspec}
  \fi
  \defaultfontfeatures{Ligatures=TeX,Scale=MatchLowercase}
\fi
% use upquote if available, for straight quotes in verbatim environments
\IfFileExists{upquote.sty}{\usepackage{upquote}}{}
% use microtype if available
\IfFileExists{microtype.sty}{%
\usepackage{microtype}
\UseMicrotypeSet[protrusion]{basicmath} % disable protrusion for tt fonts
}{}
\usepackage[margin=1in]{geometry}
\usepackage{hyperref}
\hypersetup{unicode=true,
            pdftitle={Calculus Chain Rule},
            pdfborder={0 0 0},
            breaklinks=true}
\urlstyle{same}  % don't use monospace font for urls
\usepackage{graphicx,grffile}
\makeatletter
\def\maxwidth{\ifdim\Gin@nat@width>\linewidth\linewidth\else\Gin@nat@width\fi}
\def\maxheight{\ifdim\Gin@nat@height>\textheight\textheight\else\Gin@nat@height\fi}
\makeatother
% Scale images if necessary, so that they will not overflow the page
% margins by default, and it is still possible to overwrite the defaults
% using explicit options in \includegraphics[width, height, ...]{}
\setkeys{Gin}{width=\maxwidth,height=\maxheight,keepaspectratio}
\IfFileExists{parskip.sty}{%
\usepackage{parskip}
}{% else
\setlength{\parindent}{0pt}
\setlength{\parskip}{6pt plus 2pt minus 1pt}
}
\setlength{\emergencystretch}{3em}  % prevent overfull lines
\providecommand{\tightlist}{%
  \setlength{\itemsep}{0pt}\setlength{\parskip}{0pt}}
\setcounter{secnumdepth}{0}
% Redefines (sub)paragraphs to behave more like sections
\ifx\paragraph\undefined\else
\let\oldparagraph\paragraph
\renewcommand{\paragraph}[1]{\oldparagraph{#1}\mbox{}}
\fi
\ifx\subparagraph\undefined\else
\let\oldsubparagraph\subparagraph
\renewcommand{\subparagraph}[1]{\oldsubparagraph{#1}\mbox{}}
\fi

%%% Use protect on footnotes to avoid problems with footnotes in titles
\let\rmarkdownfootnote\footnote%
\def\footnote{\protect\rmarkdownfootnote}

%%% Change title format to be more compact
\usepackage{titling}

% Create subtitle command for use in maketitle
\providecommand{\subtitle}[1]{
  \posttitle{
    \begin{center}\large#1\end{center}
    }
}

\setlength{\droptitle}{-2em}

  \title{Calculus Chain Rule}
    \pretitle{\vspace{\droptitle}\centering\huge}
  \posttitle{\par}
    \author{}
    \preauthor{}\postauthor{}
    \date{}
    \predate{}\postdate{}
  

\begin{document}
\maketitle

~

Image Source

~

This post will be about the chain rule. The chain rule was one of those
topics that took a bit of time for me to understand when I was a younger
math student. It is assumed that the reader knows about the product
rule.

~

\hypertarget{a-motivating-example}{%
\subsubsection{A Motivating Example}\label{a-motivating-example}}

~

Consider a simple function such as \(f(x) = x^3\). The derivative would
be simply \(f'(x) = 3x^2\).

But what if was expressed as \(f'(x) = 3x^2 * 1\)? Where did this 1 come
from? Let's try this:

~

\[f'(x) = 3x^2 \dfrac{d}{dx} x\]

~

The 1came from the derivative of x with respect to x.

So what did we do above? We took the derivative of \(x^3\) and then
multiplied it by the derivative of \(x\).

~

\hypertarget{the-chain-rule}{%
\subsubsection{The Chain Rule}\label{the-chain-rule}}

~

Given a (continuous) function \(h(x) = f(g(x))\) where and \(f(x)\) and
\(g(x)\) are different (continuous) functions. Then

~

\[\displaystyle h'(x) = f'( g(x) ) g'(x)\].

~

This means we take the derivative of the outside function \(f(g(x)\) and
then take the derivative of the inside function \(g(x)\). It can be
possible that the function inside \(g(x)\) can be a different function
such as \(k(x)\) which is different from \(x\).

~

\hypertarget{examples}{%
\subsubsection{Examples}\label{examples}}

~

\textbf{Example 1}

~

The function \(x^3\) from earlier has \(g(x) = x\), \(f(x) = x^3\),
\(f'(x) = 3x^2\) and \(g'(x) = 1\). The derivative of \(x^3\) is simply
\(3x^2\).

~

\textbf{Example 2}

~

Consider the function \(h(x) = \sin(2x)\). The outside function is
\(f(x) = \sin(x)\) with \(f'(x) = \cos(x)\). The inside function is
\(g(x) = 2x\) with \(g'(x) = 2\).

By Chain Rule, the derivative \(h'(x)\) is \(2 \cos(2x)\).

~

\textbf{Example 3 (Combining with Product Rule)}

~

Suppose that we have \(h(x) = x e^{x^2}\). Through product rule and
chain rule on the derivative of \(e^{x^2}\) gives:

~

\[\begin{array}{lcl} h'(x) & = & e^{x^2} + x \times e^{x^2} \times 2x \\ & = & e^{x^2} (1 + 2x^2) \\ \end{array}\]

~

\textbf{Example 4 (Multiple Chain Rule)}

~

There are cases when you may have to use multiple chain rules along with
product rules, quotient rules and so on.

Consider \(h(x) = \cos((2x + 1)^2)\). The derivative \(h'(x)\) is:

~

\[\begin{array}{lcl} h'(x) & = & - \sin(x^2) \dfrac{d}{dx} (2x + 1)^{2} \\ & = & - \sin(x^2) \times 2(2x + 1) \times \dfrac{d}{dx} (2x + 1) \\ & = & -2 \sin(x^2) (2x + 1) \times 2\\ & = & -4 \sin(x^2) (2x + 1) \end{array}\]
~

\hypertarget{tips-for-learning-chain-rule}{%
\subsubsection{Tips for Learning Chain
Rule}\label{tips-for-learning-chain-rule}}

~

Take it one step at a time.

Identify the outside function(s) and inside function(s). Start from the
outside to the inside.

Practice with the simple functions such as \(x\), \(\cos(x)\),
\(\dfrac{1}{x}\) and so on.


\end{document}
