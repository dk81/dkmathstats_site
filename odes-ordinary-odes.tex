\documentclass[]{article}
\usepackage{lmodern}
\usepackage{amssymb,amsmath}
\usepackage{ifxetex,ifluatex}
\usepackage{fixltx2e} % provides \textsubscript
\ifnum 0\ifxetex 1\fi\ifluatex 1\fi=0 % if pdftex
  \usepackage[T1]{fontenc}
  \usepackage[utf8]{inputenc}
\else % if luatex or xelatex
  \ifxetex
    \usepackage{mathspec}
  \else
    \usepackage{fontspec}
  \fi
  \defaultfontfeatures{Ligatures=TeX,Scale=MatchLowercase}
\fi
% use upquote if available, for straight quotes in verbatim environments
\IfFileExists{upquote.sty}{\usepackage{upquote}}{}
% use microtype if available
\IfFileExists{microtype.sty}{%
\usepackage{microtype}
\UseMicrotypeSet[protrusion]{basicmath} % disable protrusion for tt fonts
}{}
\usepackage[margin=1in]{geometry}
\usepackage{hyperref}
\hypersetup{unicode=true,
            pdftitle={Solving Linear Ordinary Differential Equations},
            pdfborder={0 0 0},
            breaklinks=true}
\urlstyle{same}  % don't use monospace font for urls
\usepackage{graphicx,grffile}
\makeatletter
\def\maxwidth{\ifdim\Gin@nat@width>\linewidth\linewidth\else\Gin@nat@width\fi}
\def\maxheight{\ifdim\Gin@nat@height>\textheight\textheight\else\Gin@nat@height\fi}
\makeatother
% Scale images if necessary, so that they will not overflow the page
% margins by default, and it is still possible to overwrite the defaults
% using explicit options in \includegraphics[width, height, ...]{}
\setkeys{Gin}{width=\maxwidth,height=\maxheight,keepaspectratio}
\IfFileExists{parskip.sty}{%
\usepackage{parskip}
}{% else
\setlength{\parindent}{0pt}
\setlength{\parskip}{6pt plus 2pt minus 1pt}
}
\setlength{\emergencystretch}{3em}  % prevent overfull lines
\providecommand{\tightlist}{%
  \setlength{\itemsep}{0pt}\setlength{\parskip}{0pt}}
\setcounter{secnumdepth}{0}
% Redefines (sub)paragraphs to behave more like sections
\ifx\paragraph\undefined\else
\let\oldparagraph\paragraph
\renewcommand{\paragraph}[1]{\oldparagraph{#1}\mbox{}}
\fi
\ifx\subparagraph\undefined\else
\let\oldsubparagraph\subparagraph
\renewcommand{\subparagraph}[1]{\oldsubparagraph{#1}\mbox{}}
\fi

%%% Use protect on footnotes to avoid problems with footnotes in titles
\let\rmarkdownfootnote\footnote%
\def\footnote{\protect\rmarkdownfootnote}

%%% Change title format to be more compact
\usepackage{titling}

% Create subtitle command for use in maketitle
\newcommand{\subtitle}[1]{
  \posttitle{
    \begin{center}\large#1\end{center}
    }
}

\setlength{\droptitle}{-2em}

  \title{Solving Linear Ordinary Differential Equations}
    \pretitle{\vspace{\droptitle}\centering\huge}
  \posttitle{\par}
    \author{}
    \preauthor{}\postauthor{}
    \date{}
    \predate{}\postdate{}
  

\begin{document}
\maketitle

~

Hello. This post will be a short guide on solving first order linear
ordinary differential equations (ODEs). It is assumed that the reader
knows derivatives and integrals from calculus. Knowing product rule
helps here.

~

\section{Topics}\label{topics}

~

\begin{itemize}
\tightlist
\item
  The Linear Differential Equation \& Theory
\item
  What Is The Integrating Factor \(\mu(x)\)?
\item
  The Method for Solving Linear Equations
\item
  Example
\item
  Notes
\end{itemize}

~

\subsubsection{The Linear Differential Equation \&
Theory}\label{the-linear-differential-equation-theory}

~

We start with a first order linear differential equation of the form:

~

\[\displaystyle a_1(x) \, \dfrac{dy}{dx} + a_0(x) \, y = b(x)\] ~

Dividing both sides (all terms) by \(a_1(x)\) gives us:

~

\[\displaystyle \dfrac{dy}{dx} + \dfrac{a_0(x)}{a_1(x)} \, y = \dfrac{b(x)}{a_1(x)}\]

~

Setting \(\dfrac{a_0(x)}{a_1(x)}\) as \(P(x)\) and setting
\(\dfrac{b(x)}{a_1(x)}\) as \(Q(x)\) gives us the standard form as
follows:

~

\[\text{ (1)} \displaystyle \dfrac{dy}{dx} + P(x) \, y = Q(x)\]

~

Ultimately, we want to solve for \(y(x)\) from the standard form. The
procedure in doing so is not as simple as solving separable differential
equations. The procedure here will require some ``clever'' calculus.

From (1), we multiply both sides by an integrating factor \(\mu(x)\)
(What \(\mu(x)\) is will be explained later.):

~

\[\text{ (2)}\displaystyle \mu(x) \dfrac{dy}{dx} + \mu(x) P(x) \, y = \mu(x) Q(x)\]

~

Recall that the calculus product rule is
\(\dfrac{d}{dx}(f(x) \, g(x)) = f'(x) \, g(x) + f(x) \, g'(x)\).

By setting \(\mu(x) P(x)\) as \(\mu'(x)\) and recognizing that
\(\mu(x) \dfrac{dy}{dx} + \mu'(x) \, y = \dfrac{d}{dx}(\mu(x) \, y)\),
equation (2) can be expressed as:

~

\[\text{ (3)} \displaystyle \dfrac{d}{dx}( \mu(x) \, y ) = \mu(x) Q(x)\]

~

Taking the indefinite integral on both sides with respect to the
variable \(x\) in (3) gives:

~

\[\text{ (4)}\displaystyle \mu(x) \, y = \int \mu(x) Q(x) dx + C\]

~

The solution \(y = y(x)\) to the standard from (1) above is:

~

\[\text{ (5)}\displaystyle y(x) = \dfrac{1}{\mu(x)} \, \int \mu(x) Q(x) \, dx + C\]
~

with \(C\) as the constant of integration.

~

\subsubsection{\texorpdfstring{What Is The Integrating Factor
\(\mu(x)\)?}{What Is The Integrating Factor \textbackslash{}mu(x)?}}\label{what-is-the-integrating-factor-mux}

~

We have solved for \(y(x)\) but what exactly is \(\mu(x)\) when we
multiplied both sides of the standard form? Let us investigate.

Before we obtained equation (3) above, we had \(\mu(x) P(x) = \mu'(x)\)
. This is a separable differential equation. We solve for \(\mu(x)\):

Source:
\url{http://quicklatex.com/cache3/43/ql_13b732a74dc29394e28c2436d96fce43_l3.png}
Since the integral of \(P(x)\) contains a constant anyways, we do not
need the \(C\) constant. The integration factor is
\(\mu(x) = \text{exp}(\int P(x) \, dx)\). (Recall that
\(\text{exp}(x) = e^x.)\)

~

\subsubsection{The Method for Solving Linear
Equations}\label{the-method-for-solving-linear-equations}

~

The above was the theory of solving for \(y(x)\). What if you don't care
too much about the theory and just want the procedure?

~

Here is the method:

Step a) Make sure when dealing with a first order differential equation,
you have the equation in standard form as follows:

~

\[\displaystyle \dfrac{dy}{dx} + P(x) \, y = Q(x)\]

~

Step b)

Multiply both sides (all terms) by the integration factor
\(\mu(x) = \text{exp}(\int P(x) \, dx)\):

~

\[\displaystyle \mu(x) \dfrac{dy}{dx} + \mu(x) P(x) \, y = \mu(x) Q(x)\]

~

Step c)

With \(\mu(x) P(x) = \mu'(x)\) and the left side being the result of a
product rule of \(\dfrac{d}{dx}[ \mu(x) \, y]\). Rewrite the above
equation as:

~

\[\displaystyle \dfrac{d}{dx}[ \mu(x) \, y] = \mu(x) Q(x)\] ~

Step d)

Integrate both sides with respect to \(x\) and solve for \(y(x)\) to
obtain:

~

\[\displaystyle y(x) = \dfrac{1}{\mu(x)} \, \int \mu(x) Q(x) \, dx + C\]

~

Alternate Method

This alternate method involves some memorization. If you can memorize
the standard form from step a), the integrating factor \(\mu(x)\), and
the \(y(x)\) formula in step d) then you can avoid the product rule
derivation steps in b) and c).

Example

Solve for \(y(x)\) in the differential equation
\(\displaystyle x \, y'(x) - x + y(x) = 0\)

Solution:

Dividing every term by \(x\) and rearrangement gives us the standard
form a) as follows:

\(\displaystyle y'(x) + \dfrac{y(x)}{x} = 1\) We have \(P(x) = 1/x\) and
\(Q(x) = 1\). Our integrating factor \(\mu(x)\) would be
\(\text{exp}(\int 1/x \, dx) = \text{exp}( \text{ln}(x)) = x\).

We use the formula in d) to obtain the solution for \(y(x)\):

Source:
\url{http://quicklatex.com/cache3/04/ql_32fa6f74a8c1458c8b074b3595641804_l3.png}
We have the general solution to \(y(x)\). Now we find the constant \(C\)
in \(y(x)\) such that it satisfies \(x \, y'(x) - x + y(x) = 0\). (Note
that \(y'(x) = 1/2\)).

Source:
\url{http://quicklatex.com/cache3/ff/ql_37c529f1357e1fb118596b3c4fa988ff_l3.png}
With \(C = 0\), we check our solution \(y(x)\) to see if it does satisfy
\$ x , y'(x) - x + y(x) = 0\$.

Check

Source:
\url{http://quicklatex.com/cache3/67/ql_918c3c2f72178e24ac8add6b0e620567_l3.png}

~

\subsubsection{Notes}\label{notes}

~

\begin{itemize}
\item
  The example above is a simple one to illustrate how to obtain
  \(y(x)\). Other problems involve more computations.
\item
  I have used \(y\) and \(y(x)\) interchangeably.
\item
  Don't forget the constant of integration \(C\)! However, the constant
  of integration is not needed in the integration factor \(\mu(x)\).
\item
  Differential Equations is tricky as it takes a lot of work and one
  mistake can destroy everything.
\end{itemize}

~

\subsubsection{Reference}\label{reference}

~

\begin{itemize}
\item
  Nagle R.K. et al., Fundamentals of Differential Equations and Boundary
  Value Problems - 5th Edition.
\item
  The featured spiral like image is taken from
  \url{http://www.thegreatcourses.com/courses/mastering-differential-equations-the-visual-method.html}.
\end{itemize}


\end{document}
